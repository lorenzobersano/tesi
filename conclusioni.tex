\chapter{Conclusioni}

\section{Problemi aperti}

\subsection{Su Ethereum e blockchain in generale}

\textbf{Scalabilità}
Eseguire calcoli su una blockchain è troppo lento e costoso, quindi salvare e fare calcoli
su grandi quantità di dati non è fattibile. Per fare un confronto, il circuito VISA processa circa 10.000 transazioni al secondo,
Ethereum ne processa al massimo 15 al secondo, con grandi problemi quando
la rete è molto utilizzata (come nel dicembre 2017 con il fenomeno CryptoKitties)
che causano ritardo nel processamento di transazioni e aumento del costo del gas.

\textbf{Privacy}
I dati su una blockchain sono accessibili da tutti, rendendo impossibile il salvataggio e la computazione di dati sensibili.
Ciò restringe il numero di applicazioni possibili per la blockchain.

\subsection{Su uPort}

\textbf{SDK per mobile}

Al momento della scrittura di questa sezione non è ancora disponibile un SDK maturo
per l’utilizzo dei protocolli di uPort su applicazioni native Android e iOS.

\textbf{Parziale centralizzazione}

Al momento uPort utilizza ancora dei microservizi per rendere possibili alcune parti fondamentali
dell’architettura come ad esempio il server di messaging Chasqui o il server che finanzia le transazioni Sensui,
non è sono ancora decentralizzati in mancanza di alternative solide ai modelli centralizzati.

\subsection{Su BLINC}

\textbf{Centralizzazione}

A causa delle scadenze non rispettate dal team di uPort per quanto riguarda le SDK mobile e delle scadenze
che il team di BLINC doveva rispettare è stato necessario spostare il wallet degli utenti su un server centrale
invece che mantenerlo sul device dei migranti. Questo va parzialmente contro i principi di decentralizzazione
e sovranità del dato che caratterizzano la blockchain ed il progetto uPort.

\section{Possibili scenari futuri}

\subsection{Su Ethereum e blockchain in generale}

\textbf{Scalabilità}

Sono in diverse fasi di sviluppo (alcune in fasi embrionali, altre già oltre il MVP)
alcune possibili soluzioni per risolvere i problemi di scalabilità di Ethereum.

\begin{itemize}
  \item Raiden
  \item Plasma
  \item Casper
  \item Sharding
\end{itemize}

\textbf{Privacy}

Diverse startup ed aziende stanno lavorando per rendere possibile la computazione
e il salvataggio di dati privati, tra cui Keep ed Enigma: la prima avvalendosi di contenitori
off-chain di dati privati e la seconda tramite l’uso di smart contract privati.

\subsection{Su uPort}

\textbf{SDK mobile}
Lo sviluppo delle SDK procede come si vede sui repository GitHub di uPort,
quindi si raggiungerà il livello di maturazione necessario a decentralizzare
l’architettura di BLINC in relativamente poco tempo.

\textbf{Rivoluzione dell'architettura di uPort}
L’architettura uPort cambierà radicalmente di qui a poco, passando da un’astrazione
dell’identità basata su smart contract sviluppati internamente in uPort ad una architettura
basata sugli standard proposti dalla Decentralized Identity Foundation.
A differenza dell’attuale architettura dove la creazione di un’identità richiede
due transazioni (deploy del contratto Proxy tramite chiamata allo smart contract  IdentityManager
e registrazione dell’Identity Document su smart contract Registry), la registrazione
di un’identità nella nuova architettura richiede soltanto
la creazione di un account Ethereum, ed è quindi gratuita.


\subsection{Su BLINC}

\textbf{Spostamento del wallet su telefono}

Una volta che saranno rilasciate le SDK mobile di uPort si procederà a decentralizzare
l’architettura, spostando i wallet degli utenti da MongoDB al loro smartphone e di fatto
rimuovendo la necessità di avere un backend centralizzato.

\textbf{Salvataggio degli attributi delle uPort Identity su Identity Hub}

Invece di salvare gli attributi delle identità su MongoDB, l’attuale soluzione provvisoria
e centralizzata, si passerà all’utilizzo di Identity Hub che sono, come descritto
sul repository GitHub della Decentralized Identity Foundation, dei datastore che contengono
oggetti significativi per l’identità in locazioni conosciute. Ogni oggetto in un Hub è firmato
dall’identità proprietaria ed è accessibile globalmente attraverso delle API conosciute globalmente.
Il vantaggio di un Hub rispetto ad un datastore tradizionale come può essere appunto un database
Mongo è la decentralizzazione: un’identità può avere una o più istanze di Hub che sono indirizzabili
tramite un meccanismo di routing basato su URI collegati all’identificatore dell’identità.
Tutte le istanze di Hub si sincronizzano tra di loro, garantendo così la consistenza dei dati
al loro interno e permettendo al proprietario dei dati di accedervi da ovunque, anche offline.
Molto probabilmente si sfrutteranno i 3box,
ovvero l’implementazione del team di uPort della specifica degli Identity Hub.