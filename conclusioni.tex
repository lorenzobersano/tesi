\chapter{Conclusioni}

\section{Risultati ottenuti}

Nella prima versione di BLINC io ed i miei colleghi siamo riusciti a creare una versione base del
sistema di gestione di identità ed attestazioni su di esse e sui documenti
che le riguardano (salvati su storage decentralizzato IPFS) secondo l'architettura descritta
in precedenza.

In particolare ho sviluppato il sistema di creazione e gestione delle identità e di creazione e visualizzazione
di attestazioni e dichiarazioni creando delle apposite API REST consumabili da web e da mobile.

\section{Problemi aperti}

\subsection{Su Ethereum e blockchain in generale}

\subsubsection{Scalabilità}
Eseguire calcoli e salvare molti dati su una blockchain è troppo lento e costoso. 

Per quanto riguarda il salvataggio di moli di dati
importanti si ha una soluzione che consiste nel memorizzare soltanto i riferimenti crittografici
ai dati che sono poi salvati su altri tipi
di storage decentralizzato, come IPFS o Swarm\footnote{Swarm: http://theswarm.eth.show/}.

Per quanto riguarda invece la velocità di processamento delle transazioni, per fare un confronto,
il circuito VISA processa fino a 24.000 transazioni al secondo \cite{WEBSITE:10},
Ethereum ne processa al massimo 15 al secondo, con grandi problemi quando
la rete è molto utilizzata (come nel dicembre 2017 con il fenomeno CryptoKitties \cite{WEBSITE:9})
che causano ritardo nel processamento di transazioni e aumento del costo del gas.

\subsubsection{Privacy}

La blockchain garantisce immutabilità, correttezza e trasparenza dei dati e delle computazioni:
se le prime due proprietà sono ben accette in tutti i casi in cui la blockchain è un valore aggiunto,
l'ultima è più una criticità che un punto di forza in molte applicazioni
che hanno a che fare con dati sensibili, come ad esempio applicazioni sanitarie.

\subsection{Su uPort}

\subsubsection{SDK per mobile}

Al momento sono in fase di sviluppo SDK sia per Android che per iOS, ma al momento in cui si doveva 
implementare una prima versione dell'architettura di BLINC non erano abbastanza mature per le nostre necessità (
non permettevano ad esempio di poter interagire con una rete Ethereum privata ma soltanto con le testnet pubbliche)
e soprattutto non era ancora disponibile l'SDK iOS, una tegola insormontabile per il progetto in quanto è necessario
sviluppare una app per entrambi i maggiori sistemi operativi mobile.

\subsubsection{Parziale centralizzazione}

Al momento uPort utilizza ancora dei microservizi centralizzati per rendere possibili alcune parti fondamentali
dell’architettura come il server di messaging Chasqui o il server che finanzia le transazioni Sensui.

\subsection{Su BLINC}

\subsubsection{Centralizzazione}

A causa delle scadenze non rispettate dal team di uPort per quanto riguarda le SDK mobile e delle scadenze
che il team di BLINC doveva rispettare è stato necessario spostare il wallet degli utenti su un server centrale
invece che mantenerlo sul device dei migranti. Questo va parzialmente contro i principi di decentralizzazione
e sovranità del dato che caratterizzano la blockchain ed il progetto uPort.

\subsubsection{Utilizzo di librerie e protocolli uPort datati o deprecati}

Dato che nella prima versione di BLINC non è stato possibile spostare il wallet degli utenti sul loro telefono
a causa dell'immaturità degli SDK mobile,
è stato necessario spostare la gestione dell'identità sul server.

Questa necessità, oltre a quella di dover creare le identità su una blockchain privata, ha precluso l'utilizzo di molte
librerie (che erano anche quelle più supportate) di uPort, portandomi di fatto a dover utilizzare l'unica che soddisfacesse
i nostri requisiti, ovvero la \texttt{uport-js-client} già citata in precedenza.

Usando la libreria sono stati però riscontrati diversi problemi:

\begin{itemize}
  \item Errori strutturali e sintattici del codice: utilizzo di alcune variabili non dichiarate
  e errori nell'esecuzione di alcune funzioni, sistemati velocemente.
  \item Utilizzo di URI deprecati: gli URI di richiesta di informazioni o di creazione di attestazioni utilizzati nella libreria erano
  diversi da quelli specificati nella documentazione di uPort e deprecati da tempo.
  Dato che l'utilizzo è solo temporaneo per il progetto ho preferito mantenere gli URI
  deprecati per non perdere troppo tempo a rifattorizzare l'intera libreria.
  \item La memorizzazione delle attestazioni non veniva fatta \emph{on chain}: 
  come da requisiti, le attestazioni sugli utenti devono essere immutabili e quindi salvate su blockchain, ma la versione iniziale di
  \texttt{uport-js-client} non permetteva ciò, come si vede nel codice seguente,
  che va semplicemente ad aggiungere le attestazioni come campi di un oggetto
  \texttt{UPortClient}:

  \begin{lstlisting}[language=JavaScript, numbers=none]
addAttestationRequestHandler(uri) {
  const params = getUrlParams(uri)

  // Gets the JWT from the URI
  const attestations = Array.isArray(params.attestations) ? params.attestations : [params.attestations]

  for (let jwt in attestations) {
    jwt = attestations[jwt]

    // Decodes the JWT passed in the URI and gets the payload part
    const json = decodeToken(jwt).payload
    const key = Object.keys(json.claim)[0]

    if (this.network) {
      this.verifyJWT(jwt).then(() => {

        // This just adds attestations to the UportClient object, while we want attestations to be stored on the blockchain
        this.credentials[key] ? this.credentials[key].append({jwt, json}) : this.credentials[key] = [{jwt, json}]
      }).catch(console.log)
    }
  }
}
  \end{lstlisting}

  Ciò che si vuole per BLINC è però che tutte le attestazioni siano visibili a tutti
  e approvate su blockchain, perciò ho dovuto effettuare delle modifiche direttamente
  alla libreria per inserire correttamente attestazioni 
  sull'\texttt{ethereum-claims-registry}:

  \begin{lstlisting}[language=JavaScript, numbers=none]
async addAttestationRequestHandler(uri) {

  // Creates an object contract with a specific ABI 
  // and at a specific address for the EthereumClaimsRegistry contract
  const ClaimsReg = Contract(EthereumClaimsRegistryArtifact.abi).at(
    this.network.claimsRegistry
  );

  const params = getUrlParams(uri);
  const attestations = Array.isArray(params.attestations)
    ? params.attestations
    : [params.attestations];

  let i = 0;

  return Promise.all(

    // This method also adds the possibility to add more than one attestation a time
    attestations.map(async (jwt, i) => {
      const json = decodeToken(jwt).payload;
      const issAddress = json.iss;
      const subAddress = json.sub;
      const key = Object.keys(json.claim)[i];
      const value = Object.values(json.claim)[i];

      // Creates the transaction URI that will be given to the consume function
      // which will call send a transaction to the EthereumClaimsRegistry contract on the private blockchain
      // with the parameters passed in the JWT
      const tx = ClaimsReg.setClaim(subAddress, key, value);

      try {
        const txHash = await this.consume(tx);
        const receipt = await this.getReceipt(txHash);
        const log = receipt.logs[0];

        const claimSetEventAbi = ClaimsReg.abi.filter(
          obj => obj.type === 'event' && obj.name === 'ClaimSet'
        )[0];


        // Gets the ClaimSet event emitted by the setClaim function, to give feedback to the NodeJS backend
        // that the claims were correctly set
        const decodedEvent = decodeEvent(
          claimSetEventAbi,
          log.data,
          log.topics
        );

        // Returns an object which is added to the response array
        return {
          key: ethutil.toUtf8(decodedEvent.key),
          value: ethutil.toUtf8(decodedEvent.value),
          issuer: decodedEvent.issuer,
          sub: decodedEvent.subject
        };
      } catch (error) {
        return Promise.reject(error);
      }
    })
  );
}
  \end{lstlisting}
\end{itemize}

\section{Possibili scenari futuri}

\subsection{Su Ethereum e blockchain in generale}

\subsubsection{Scalabilità}

Sono in diverse fasi di sviluppo (alcune in fasi embrionali, altre già oltre il MVP)
alcune possibili soluzioni per risolvere i problemi di scalabilità di Ethereum.

\begin{itemize}
  \item Raiden (state channels) \footnote{Raiden: https://raiden.network/}
  \item Plasma (side-chains) \footnote{Plasma: http://plasma.io/}
  \item Casper (Proof-of-Stake) + Sharding \footnote{Sharding: https://github.com/ethereum/wiki/wiki/Sharding-FAQs}
\end{itemize}

\subsubsection{Privacy}

Diverse aziende stanno lavorando per rendere possibile la computazione
e il salvataggio di dati privati, tra cui Keep\footnote{https://keep.network/} \cite{ARTICLE:1} ed Enigma\footnote{https://enigma.co/}:
la prima avvalendosi di contenitori off-chain di dati privati e la seconda tramite
l’uso di smart contract privati \cite{ARTICLE:2}.

\subsection{Su uPort}

\subsubsection{SDK mobile}

Lo sviluppo delle SDK procede come si vede sui repository GitHub di uPort,
quindi si raggiungerà il livello di maturità necessario a decentralizzare
l’architettura di BLINC in relativamente poco tempo.

\subsubsection{Rivoluzione dell'architettura di uPort}

L’architettura uPort cambierà radicalmente di qui a poco, passando da un’astrazione
dell’identità basata su smart contract sviluppati internamente in uPort ad una architettura
basata sugli standard proposti dalla Decentralized Identity Foundation.

A differenza dell’attuale architettura dove la creazione di un’identità richiede
due transazioni (deploy del contratto Proxy tramite chiamata allo smart contract IdentityManager
e registrazione dell’Identity Document su smart contract Registry), la registrazione
di un’identità nella nuova architettura richiederà soltanto
la creazione di un account Ethereum, ed è quindi gratuita, dettaglio non da poco perchè
permetterà di rimuovere il loro server di finanziamento delle transazioni Sensui, andando di fatto ad
avvicinarsi ad una architettura per la gestione di identità completamente decentralizzata.

\subsection{Su BLINC}

\subsubsection{Spostamento del wallet su telefono}

Una volta che saranno rilasciate le SDK mobile di uPort si procederà a decentralizzare
l’architettura, spostando i wallet degli utenti da MongoDB al loro smartphone e di fatto
rimuovendo la necessità di avere un backend centralizzato.

\subsubsection{Salvataggio degli attributi delle uPort Identity su Identity Hub}

Invece di salvare gli attributi privati e non delle identità su MongoDB, l’attuale soluzione provvisoria
e centralizzata, si passerà all’utilizzo di Identity Hub che sono, come descritto
sul repository GitHub della Decentralized Identity Foundation, dei datastore che contengono
oggetti significativi per l’identità in locazioni conosciute. Ogni oggetto in un Hub è firmato
dall’identità proprietaria ed è accessibile globalmente attraverso delle API conosciute globalmente.
Il vantaggio di un Hub rispetto ad un datastore tradizionale come può essere appunto un database
Mongo è la decentralizzazione: un’identità può avere una o più istanze di Hub che sono indirizzabili
tramite un meccanismo di routing basato su URI collegati all’identificatore dell’identità.
Tutte le istanze di Hub si sincronizzano tra di loro, garantendo così la consistenza dei dati
al loro interno e permettendo al proprietario dei dati di accedervi da ovunque, anche offline.
Molto probabilmente si sfrutterà 3Box\footnote{3Box: http://alpha.3box.io/},
ovvero l’implementazione del team di uPort della specifica degli Identity Hub.